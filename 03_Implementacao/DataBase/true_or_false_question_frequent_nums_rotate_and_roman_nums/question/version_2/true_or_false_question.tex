\documentclass[12pt,varwidth=16cm,border=1pt]{standalone}

% este documento é um template.

% este documento é usado pelo script make_random_versions.py para
% criar as versões desta pergunta

% pacote usado para mostrar listagens de programas
\usepackage{listings}
\renewcommand{\lstlistingname}{Listagem}
\lstset{
  language=python,
  frame=single,
}

\begin{document}

Considere o programa, Pyhton 3, que se segue. Implemente a função \verb+most_frequent()+, que devolver o número mais frequente de uma dada lista, a função \verb+least_frequent()+, que devolve o número menos frequente de uma dada lista. Aconselha-se a exploração das funções Python \verb+min()+ e \verb+max()+. Implemente a função \verb+rotate_list()+, que roda os elementos de uma lista em torno do índice recebido no parâmetro \textit{k} e finalmente a função \verb+int_to_roman+, que transforma um número decimal em numeração romana, para esta função assuma que o output de um número negativo é igual ao do número positivo.

\lstinputlisting{program2.py}

Execute o seguinte código para testar o funcionamento das funções criadas.

\lstinputlisting{program3.py}

Acrescente a este programa o código que lhe permita indicar se as
afirmações seguintes são verdadeiras ou falsas.

Indique se é verdadeiro ou falso.

\end{document}
