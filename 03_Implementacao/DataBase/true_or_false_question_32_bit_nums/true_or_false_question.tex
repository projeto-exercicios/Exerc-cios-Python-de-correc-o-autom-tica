\documentclass[12pt,varwidth=16cm,border=1pt]{standalone}

% este documento é um template.

% este documento é usado pelo script make_random_versions.py para
% criar as versões desta pergunta

% pacote usado para mostrar listagens de programas
\usepackage{listings}
\renewcommand{\lstlistingname}{Listagem}
\lstset{
  language=python,
  frame=single,
  basicstyle=\ttfamily\footnotesize,
}

\begin{document}

Considere o programa, Python 3, que se segue. Escreva a classe \verb+Reverser+. Os objetos da classe \verb+Reverser+ tem o método \verb+reverse()+. O método \verb+reverse()+ reverte todos os números que não possuam mais de 32 bits, ou seja, os números positivo com mais de 32 bits irão reverter para números negativos e vice-versa. 

Escreva ainda as funções \verb+to_bits()+ e \verb+binary_sum()+, estas não são métodos da classe acima. A função \verb+to_bits()+ converte um número decimal em binário e a função \verb+binary_sum()+ soma dois números binários.

\lstinputlisting{program2.py}

Considere o código de testes, à classe \verb+Reverser+ e funções \verb+to_bits()+ e \verb+binary_sum()+, que se segue.

\lstinputlisting{program3.py}

Acrescente a este programa o código que lhe permita indicar se as
afirmações seguintes são verdadeiras ou falsas.

Indique se é verdadeiro ou falso.

\end{document}
