\documentclass[12pt,varwidth=16cm,border=1pt]{standalone}

% este documento é um template.

% este documento é usado pelo script make_random_versions.py para
% criar as versões desta pergunta

% pacote usado para mostrar listagens de programas
\usepackage{listings}
\renewcommand{\lstlistingname}{Listagem}
\lstset{
  language=python,
  frame=single,
}

\begin{document}

Considere o programa, Pyhton 3, que se segue. Neste exercício é apresentada a classe numpy e algumas fácilidades que esta apresenta perante a geração e manipulação de arrays n-dimensionais. Implemente a função \verb+join_arrays(dimension, arr_1, arr_2)+, esta recebe uma dimensão e 2 arrays, e retorna a junção dos dois, caso a dimensão for diferente da suportada, é devolvido um array aleatório, caso seja necessário esta função deve de igualar o tamanho dos arrays recebido, preenchendo o array de menor dimensão com "1's".

\lstinputlisting{program2.py}

Acrescente a este programa o código que lhe permita indicar se as
afirmações seguintes são verdadeiras ou falsas.

Indique se é verdadeiro ou falso.

\end{document}
