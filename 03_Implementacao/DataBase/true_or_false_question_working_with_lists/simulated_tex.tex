\documentclass[12pt,varwidth=16cm,border=1pt]{standalone}

% este documento é um template.

% este documento é usado pelo script make_random_versions.py para
% criar as versões desta pergunta

% pacote usado para mostrar listagens de programas
\usepackage{listings}
\renewcommand{\lstlistingname}{Listagem}
\lstset{
  language=python,
  frame=single,
}

\begin{document}

Considere o programa, Pyhton <b><font color=red><i>3</font></i></b>, que se segue. A função \verb+random\_string\_generator()+ gera uma string com caracteres aleatórios, o segundo ciclo \textit{for} do código apresentado adiciona em cada iteração um valor inteiro a lista <b><font color=red><i>\verb+o+</font></i></b> e uma string a lista <b><font color=red><i>\verb+l+</font></i></b>, o terceiro ciclo \textit{for} a cada iteração multiplica o elemento da lista <b><font color=red><i>\verb+o+</font></i></b> por <b><font color=red><i><b><font color=red><i>\verb+26+</font></i></b></font></i></b> .

\lstinputlisting{program.py}

Acrescente a este programa o código que lhe permita indicar se as
afirmações seguintes são verdadeiras ou falsas.

Indique se é verdadeiro ou falso.

\end{document}
