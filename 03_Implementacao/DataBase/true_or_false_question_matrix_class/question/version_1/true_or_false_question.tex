\documentclass[12pt,varwidth=16cm,border=1pt]{standalone}

% este documento é um template.

% este documento é usado pelo script make_random_versions.py para
% criar as versões desta pergunta

% pacote usado para mostrar listagens de programas
\usepackage{listings}
\renewcommand{\lstlistingname}{Listagem}
\lstset{
	language=python,
	frame=single,
}

\begin{document}
	
Considere o programa, Pyhton 3, que se segue. A classe M dispõe do método \verb+__init__()+ que inicia um objeto da classe do tipo matriz e o método \verb+__repr__()+ que faz print da matriz. Crie o método \verb+row_times_column()+, que multiplica uma linha por uma coluna e vice-versa, recebe como parâmetros o índice da linha e coluna em questão, o método \verb+matrix_cross_product()+, que devolve o produto cruzado de uma matriz, o método \verb+std()+, que calcula o desvio padrão da matriz e finalmente o método \verb+mult_by_itself()+ que multiplica a matriz por si mesma.

\lstinputlisting{program2.py}

Para testar o funcionamento das funções execute o seguinte código.

\lstinputlisting{program3.py}

Acrescente a este programa o código que lhe permita indicar se as
afirmações seguintes são verdadeiras ou falsas.

Indique se é verdadeiro ou falso.
	
\end{document}
