\documentclass[12pt,varwidth=16cm,border=1pt]{standalone}

% este documento é um template.

% este documento é usado pelo script make_random_versions.py para
% criar as versões desta pergunta

% pacote usado para mostrar listagens de programas
\usepackage{listings}
\renewcommand{\lstlistingname}{Listagem}
\lstset{
  language=python,
  frame=single,
}

\begin{document}

Considere o programa, Pyhton 3, que se segue. A lista <b><font color=red><i>\verb+n+</font></i></b> é constituída por pixeis RGB, simulando uma imagem. Para responder às questões seguintes acrescente ao código do enunciado código que permita guardar em <b><font color=red><i>\verb+o+</font></i></b> a conversão da imagem original para tons de cinzento e guardar em <b><font color=red><i>\verb+k+</font></i></b> a imagens em tons de cinzento aplicado um filtro de \verb+Treshold+ com um limiar de <b><font color=red><i>1338344</font></i></b>. Para converter a imagem em tons de cinzento use a equação \verb+0.2989 * R + \verb+0.5870 * G + \verb+0.1140 * B+, onde R, G e B são \textit{red}, \textit{green} e \textit{blue}, arredonde o resultado a duas casas decimais usando a função built-in \verb+round+

\lstinputlisting{program.py}

Acrescente a este programa o código que lhe permita indicar se as
afirmações seguintes são verdadeiras ou falsas.

Indique se é verdadeiro ou falso.

\end{document}
